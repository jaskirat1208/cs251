\documentclass{article}
\usepackage[utf8]{inputenc}
\usepackage{amsmath,amsthm,amssymb}
\usepackage[linesnumbered,vlined,ruled]{algorithm2e}
\usepackage{graphicx}
\title{An algorithm for multiplying two digit numbers}
\author{Jaskirat Singh }
\date{3 March 2017}

\begin{document}

\maketitle

\section{Introduction}
    This is an algorithm for multiplying two digit numbers mentally and within less time.
    This algorithm considers two specific cases: 
    \begin{itemize}
        \item The first digits are same, and the last ones add up to 10
        \item The last digits are same, and the first ones add up to 10
    \end{itemize}
    Ex: %{ \hspace{10mm} %\space 
    \begin{itemize}
        \item Case 1:
            \begin{tabular}{c c c}
                &6 &4\\
                x &6 &6 \\
                \hline
                & 6(6+1) & 4*6 \\
                & 42     & 24  \\
            \end{tabular}
            \\ Ans: 4224
        
        \item Case 2:
            \begin{tabular}{c c c}
                &3 &4\\
                x &7 &4 \\
                \hline
                & 3*7+4 & 4*4 \\
                & 25     & 16  \\
            \end{tabular}
            \\
            \\
             Ans: 2516            
    \end{itemize}
\section{How to mentally implement it ? }
    Implementation is really simple for these two cases. Just do the following:
    \begin{itemize}
        \item Case 1:
            \begin{itemize}
                \item First two digits : just add the first digit to its square
                \item Last two digits  : multiply the last to digits
            \end{itemize}
        \item Case 2:
            \begin{itemize}
                \item First two digits : multiply the first two digits and add the last digit to the product.
                \item Last two digits  : just square the last digit
            \end{itemize}
        
    \end{itemize}
    \floatsep 1cm
    \includegraphics[width=5cm, height=4cm ]{image} \ \ 
    \includegraphics[width=5cm, height=4cm]{img2}
\section{Mathematical Derivation of the algorithm}
   
    %Now,
%    \begin{ams}
    \begin{proof}
         Let the numbers be $A$ and $B$  where 
         $A = 10*a_{1} + a_{2}$ and 
         $B = 10*b_{1} + b_{2}$     
         \\
         \begin{equation}   \label{main}
             A*B = 100.a_{1}.a_{2}+10.(a_{1}.b_{2}+a_{2}.b_{1}) + b_{1}.b_{2}
         \end{equation} \\
         
        \begin{itemize}
                 \item  Case 1: $a_{1} = b_{1}$ and $a_{2}+ b_{2}=10$\\ 
                        \begin{proof}
                                                    \begin{align*}
                                a_{1}.b_{2} &+ a_{2}.b_{1} = 10.a_{1} \\
                                %&& \text{Thus,}
                                From\  \ref{main},\ \ \ \ \ \ \  &\\
                                A.B &=100.a_{1}.a_{2}+10.(a_{1}.b_{2}+a_{2}.b_{1}) + b_{1}.b_{2}  \\
                                &=100.{a_{1}.b_{1}} + 100.a_{1} + a_{2}.b_{2}\\
                                &=100(a_{1}.b_{1} + a_{1}) + a_{2}.b_{2} \\
                                &=100.a_{1}.(a_{1} + 1) + a_{2}.b_{2}
                        \end{align*}
                        \end{proof}
                 \item  Case 2: $a_{2} = b_{2}$ and $a_{1}+ b_{1}=10$\\
                    \begin{proof}    
                                                \begin{align*}
                                a_{1}.b_{2} &+ a_{2}.b_{1} = 10.a_{2} \\
                                %&& \text{Thus,}
                                From\  \ref{main},\ \ \ \ \ \ \  &\\
                                A.B &=100.a_{1}.a_{2}+10.(a_{1}.b_{2}+a_{2}.b_{1}) + b_{1}.b_{2}  \\
                                &=100.{a_{1}.b_{1}} + 100.a_{2} + a_{2}.b_{2}\\
                                &=100(a_{1}.b_{1} + a_{2}) + a_{2}^2
                        \end{align*}
                    \end{proof}    
        \end{itemize}
    \end{proof}

\section{Algorithm$\footnote{\label{IMPORTANT NOTE}Note that this algorithm is fast only for these two cases.For other cases,refer the bibliography}$}
    \begin{algorithm}[]
        \caption{Multipy 2 two diit numbers A and B}
        %\label{Algorithm:delete_mCBF_aCBF}
       $a_{1} $=$ A/10$\\
       $a_{2} $=$ A \% 10$ \\
       $b_{1} $=$ B/10$\\
       $b_{2} $=$ B \% 10$\\
       \If{$a_{1} = b_{1}$ and $a_{2} +b_{2} =10$}{$left \gets a_{1}(a_{1}+1)$\\
       $right \gets a_{2}b_{2}$\\
        print $left$ $right$

       }
       \ElseIf{$a_{2} = b_{2}$ and $a_{1} +b_{1} =10$}
        {
            $left \gets a_{1}.b_{1} + a_{2}$\\
            $right \gets a_{2}^2$\\
            print $left$ $right$
        }
        
    \end{algorithm}
\section{Extension for more numbers}
    The algorithm that we have studied is valid for two digit numbers only. The next question arises, can we further extend this algo for more than two digits?\par
    The answer is YES. Analogous to the proof for two digit case, we can get the algorithm for a more than two digit case.But there will be certain restrictions.We would be able to divide the numbers into two smaller parts and then if the conditions apply , we could use that algorithm.For further details, see the bibliography.It will be clear with the following example: 125 * 175 = 21875\\
    \begin{center}
        \begin{tabular}{c c c}
              &1 & 25  \\
             x&1 & 75  \\
             \hline
             &1.(1+1) & 25*75\\
             &2 &1875
             
        \end{tabular}
    \end{center}
    Thus, the answer is 21875.\\
    % \footnote{\label{IMPORTANT NOTE}Note that this algorithm is fast only for these two cases.For other cases,refer the bibliography}
    
    All these ideas are not new, but they have their roots in atharvaveda. \\
    This shows that even in acient period, when the vedas were composed, the people of that time were aware of such techniques.\cite{atharvaveda}
    %Ex: {Case 2:} \hspace{10mm}%# \space
    
    \bibliographystyle{ieeetr}%Used BibTeX style is unsrt
    \bibliography{citations}
   
    %Ex: {Case 2:} \hspace{10mm}%# \space
    

\end{document}
